\documentclass[12pt]{report}

% Paquetes necesarios
\usepackage[utf8]{inputenc} % Codificación UTF-8
\usepackage[spanish]{babel} % Idioma español
\usepackage{graphicx}       % Para incluir imágenes
\usepackage{hyperref}       % Hipervínculos
\usepackage{geometry}       % Configurar márgenes
\usepackage{titlesec}       % Modificar títulos
\usepackage{fancyhdr}       % Encabezados y pies de página

% Configuración de márgenes
\geometry{a4paper, margin=1in}

% Configuración de hipervínculos
\hypersetup{
    colorlinks=true,
    linkcolor=blue,
    urlcolor=blue,
    pdftitle={Título de la Tesis},
    pdfauthor={Tu Nombre}
}

% Configuración del encabezado y pie de página
\pagestyle{fancy}
\fancyhf{}
\fancyhead[L]{Título de la Tesis}
\fancyhead[R]{\thepage}
\fancyfoot[C]{Nombre del Autor}

% Configuración de los títulos
\titleformat{\chapter}[hang]{\bfseries\Large}{\thechapter.}{1em}{}

% Inicio del documento
\begin{document}

% Portada
\begin{titlepage}
    \centering
    {\Large \textbf{Universidad X}}\\[2cm]
    {\Huge \textbf{Título de la Tesis}}\\[1.5cm]
    {\large Por}\\[0.5cm]
    {\Large \textbf{Tu Nombre}}\\[2cm]
    \vfill
    {\large \textbf{Fecha:} \today}
\end{titlepage}

% Índice
\tableofcontents
\newpage

% Capítulos
\chapter*{Introducción}
\addcontentsline{toc}{chapter}{Introducción}
\hypertarget{introduccion}{}
Escribe aquí la introducción de tu tesis. Explica el contexto, los objetivos y la estructura general del documento.
\setcounter{chapter}{0}
\chapter{Marco Teórico}
Incluye el marco teórico, revisiones de la literatura y conceptos clave relacionados con tu trabajo.
\section{Conceptos Fundamentales}
En esta sección se definen los conceptos clave que son relevantes para este trabajo. 

\subsection{Definición del Término A}
El término A se refiere a...

\subsection{Definición del Término B}
El término B, por otro lado, se entiende como...

\section{Teorías Relacionadas}
Esta sección presenta las teorías principales que fundamentan el estudio.
\chapter{Metodología}
Describe la metodología utilizada en tu investigación, incluyendo los métodos, herramientas y procedimientos.

\chapter{Resultados}
Presenta los resultados obtenidos durante tu investigación.

\chapter{Discusión}
Analiza los resultados y compara con otros estudios o referencias relevantes.

\chapter{Conclusiones y Recomendaciones}
Resume los hallazgos principales y ofrece recomendaciones futuras.

% Bibliografía
\addcontentsline{toc}{chapter}{Bibliografía}
\bibliographystyle{plain}
\bibliography{referencias}

% Anexos (opcional)
\appendix
\chapter{Anexo A}
Incluye aquí cualquier información adicional como encuestas, gráficos, o tablas complementarias.

\end{document}