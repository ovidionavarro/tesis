\documentclass[12pt]{report}

% Paquetes necesarios
\usepackage[utf8]{inputenc} % Codificación UTF-8
\usepackage[spanish]{babel} % Idioma español
\usepackage{graphicx}       % Para incluir imágenes
\usepackage{hyperref}       % Hipervínculos
\usepackage{geometry}       % Configurar márgenes
\usepackage{titlesec}       % Modificar títulos
\usepackage{fancyhdr}       % Encabezados y pies de página
\usepackage{gensymb}
\usepackage{textcomp}
\usepackage{upquote}
%\setlength{\parindent}{1.5em}
% Configuración de márgenes
\geometry{a4paper, margin=1in}


% Configuración de hipervínculos

\hypersetup{
    colorlinks=true,
    linkcolor=blue,
    urlcolor=blue,
    citecolor=blue,
    pdftitle={Título de la Tesis},
    pdfauthor={Tu Nombre}
}

% Configuración del encabezado y pie de página
%\pagestyle{fancy}
%\fancyhf{}
%\fancyhead[L]{Título de la Tesis}
%\fancyhead[R]{\thepage}
%\fancyfoot[C]{Nombre del Autor}

% Configuración de los títulos
\titleformat{\chapter}[hang]{\bfseries\Large}{\thechapter.}{1em}{}

% Inicio del documento
\begin{document}

% Portada
\begin{titlepage}
    \centering
    {\Large \textbf{Universidad X}}\\[2cm]
    {\Huge \textbf{Título de la Tesis}}\\[1.5cm]
    {\large Por}\\[0.5cm]
    {\Large \textbf{Tu Nombre}}\\[2cm]
   
    {\large \textbf{Fecha:} \today}
\end{titlepage}

% Índice
\tableofcontents
\newpage

% Capítulos
\chapter*{Introducción}
\addcontentsline{toc}{chapter}{Introducción}
\hypertarget{introduccion}{}


	En la actualidad, una gran mayoría de los usuarios tiende a ignorar las recomendaciones de seguridad al momento de crear contraseñas alfanuméricas. Es común observar el uso de contraseñas cortas y cargadas de información personal, lo cual facilita su memorización, pero también aumenta significativamente su vulnerabilidad frente a ataques de fuerza bruta o de diccionario [[1,4,5,6]].
	
	Para abordar estas debilidades, se han desarrollado nuevas alternativas, entre las cuales destacan las contraseñas gráficas. Este enfoque se basa en la capacidad humana de recordar patrones visuales en una imagen con mayor facilidad que largas cadenas de caracteres alfanuméricos. En este tipo de contraseñas, el usuario debe recordar una imagen o partes específicas de ella mediante la selección  de determinados puntos, imágenes que dadas sus características posean un amplio espacio para la construcción de sus contraseñas y de este modo ser más resistentes a los ataques de diccionarios, obteniendo un espacio de búsqueda mucho más amplio y resistente a los ataques comunes.
	
	En este trabajo, específicamente, vamos a trabajar sobre el PassPoint[1], una técnica de autenticación gráfica que en su fase de registro consiste en seleccionar cinco puntos de una imagen elegida por el usuario. Durante la autenticación, el usuario debe hacer clic en una determinada vecindad y en el mismo orden de los puntos registrados. Entre las debilidades de esta técnica se encuentran los Hotspots[6](puntos más probables a seleccionar por el usuario), además diversos patrones predefinidos que los usuarios tienden a seguir durante el registro para facilitar la memorización. Estos patrones incluyen formas específicas como Z, W, C, V, patrones agrupados o regulares, y patrones LOD, DIAG o LINE (formas de línea o diagonales).
	
	La tendencia de los usuarios a crear patrones entre los puntos seleccionados, ya sea de manera independiente o en combinación con Hotspots, constituye una debilidad importante. Esto hace que las contraseñas generadas sean menos aleatorias y más susceptibles a ataques basados en diccionarios específicos. Por ello, resulta fundamental desarrollar pruebas que detecten la existencia de estos patrones en las contraseñas antes de su uso, ya que contribuirían significativamente a mejorar la seguridad de la técnica PassPoint.
	
	
	
	
	
	
	
	%%%de la tesis de liset arreglar
	%Se han encontrado pocas investigaciones dirigidas en este aspecto durante los ultimos años. Tres de los test más empleados para comprobar Aleatoriedad Espacial Completa  son: el test de la funciónK-Ripley, el test de la función G o de la distancia al vecino más cercano y el test de la función F o de la distancia de espacio vacío; sin embargo, en [9-10 sensor] se demuestra que en el escenario de PassPoint dos de  estos tests resultan ser inefectivos en la detección de contraseñas gráficas formadas por patrones agrupados, mientras que en [15, 16] demuestran que los tres test son inefectivos tanto para detectar agrupamiento como regularidad en las contraseñas de este escenario.\textbf{ARREGLAR}. Hasta el momento en la bibliografia consultada se encuentran dos test efectivos(tesis lisset y joakin) para detectar contraseñas no aleatorias formadas por patrones agrupados o regulares e el escenario PassPoint, basados en (tesis de liset) en la distancia promedio de los perímetros de los triángulos de Delaunay y (joakin)en distancia media entre 5 puntos.
	A lo largo de los últimos años, se han realizado pocas investigaciones enfocadas en este tema. Entre los métodos más comunes para evaluar la Aleatoriedad Espacial Completa se encuentran: el test de la función K-Ripley, el test de la función G, que analiza la distancia al vecino más cercano, y el test de la función F, que se centra en la distancia de espacio vacío. Sin embargo, en [9-10 sensor] se demuestra que, en el contexto de PassPoint, dos de estos métodos son ineficaces para detectar contraseñas gráficas compuestas por patrones agrupados. Por otro lado, en [15, 16] se evidencia que los tres tests no logran identificar ni el agrupamiento ni la regularidad en las contraseñas de este escenario. Hasta ahora, en la bibliografía revisada, se han encontrado dos tests efectivos (tesis de Liset y Joakin) para identificar contraseñas no aleatorias que presentan patrones agrupados o regulares en el contexto de PassPoint. Estos métodos se basan, en el caso de (Liset), en la distancia promedio de los perímetros de los triángulos de Delaunay, y, en el caso de (Joakin), en la distancia media entre cinco puntos.
	
	%bibliografia de sensor
	%Teneindo en cuenta que las caracteristicas de una triangulacion de delaunay permiten extraer informacion sobre la dependencia entre puntos ,se ha utilizado como herramienta a mediados de la decada de 1980 para dectectar patrones de puntos. En[22]  Chiu utilizo varias de estas caracteristicas para detectar la agrupacion y la regularidad entre los puntos. En particular, la caracteristica del angulo maximo de un triangulo de delaynay, en la bibliografia consultada ,nunca se habia utilizado para detactar otro tipo de patrones ademas de los agrupados o regulares . Sin embargo dado que los patrones diag o line se caracterizan por tener un angulo cercano a 0 entre dos segmentos consecutivos , en [sensor] se planteo  y se demostro que la media de los angulos maximos de los triangulos de delaunay formados a partir de los puntos de las contraseñas graficas de PassPiont  ,es un estadigrafo eficas para detectar la existencia de patrones DAIG y LINE,a pesar de un numero reducido de puntos .
	Teniendo en consideración que las propiedades de una triangulación de Delaunay brindan la capacidad de obtener información acerca de la interrelación entre puntos, se ha empleado como una herramienta en la mitad de la década de 1980 para identificar configuraciones de puntos. En el estudio realizado por Chiu [22], se emplearon varias de estas propiedades para reconocer la agrupación y la regularidad entre los puntos. Específicamente, la característica del`` ángulo máximo de un triángulo de Delaunay", según la literatura revisada, nunca había sido utilizada previamente para identificar otro tipo de configuraciones además de las agrupadas o regulares. No obstante, dado que los patrones ``DIAG " y ``LINE " se distinguen por presentar un ángulo cercano a 0{\degree } entre dos segmentos consecutivos, en [sensor] se propuso y demostró que la media de los ángulos máximos de los triángulos de Delaunay generados a partir de los puntos de las contraseñas gráficas de PassPoint es un estadígrafo eficaz para detectar la presencia de patrones ``DAIG" y ``LINE", incluso con un número limitado de puntos.
	En [15] se entendía como el ángulo formado entre dos segmentos consecutivos ,el menor de los dos ángulos que forman la intersección de la prolongacin de los segmentos de una contraseña.Nos referiremos al mayor de estos dos angulos como el angulo adyacente entre dos segmentos .

	

	
%	Existen varios articulos publicados sobre el tema en los ultimos años, en (articulo legon2019)se propone un modelo probabilisto de autenticación gráfica que permite medir en la practica el nivel de autenticidad de el usuario en alto, medio y bajo ,.En (sensor) se realiza un test de deteccion de  patrones suaves  DIAG y LINE basado en el promedio angulos maximos de sus triangulos de delaunay, según los resultados obtenidos se demostró que el promedio de los angulos maximos de los triangulos de delaunay es un estadistico eficas para detectar contraseñas que sigan un patron DIAG o LINE
	
	Según los resultados obtenidos en sensor se sugiere investigar la eficacia de un test de detección de los patrones diag y line utilizando como estadigrafo el promedio de los angulos maximos de los triangulos de delaunay pero realizando un analisis independientes segun la cantidad de triangulos en su triangulacion (3,4,5) y realizar un analisis comparativo respcto a los resultados obtenidos en s(sensor).\\
	
	\large{\textbf{Problema de investigacion:}}\\
	
	\large{\textbf{Objetivo de estudio:}}\\
	
	\large{\textbf{Campo de accion:}}\\
	
	\large{\textbf{Hipotesis:}}\\
	
	\large{\textbf{Idea de la solucion:}}\\
	
	\large{\textbf{Objetivos}}\\
	\normalsize{\textbf{Objetivos generales}}:Detectar\cite{1} la no aleatoriedad en las constraseñas graficas que siguen patrones DIAG o LINE en PassPoint para cada numero de triangulos en una triangulacion de delaunay\\
	\normalsize{\textbf{Objetivos especificos}}:
	\begin{itemize}
		\item Crear un criterio de desicion basado en el promedio de los angulos maximos de los triangulos de delaunay para cada posible cantidad de triangulos en su triangulacion
		\item  Para cada cada cantidad de triangulos, obtener stimaciones treoricas del numero esperado de fallos en la deteccion de contraseñas graficas debiles
	\end{itemize}
	
	
	

%%buscar en tesis de lisset los tipos de ataques
\setcounter{chapter}{0}
\chapter{Marco Teórico}
Incluye el marco teórico, revisiones de la literatura y conceptos clave relacionados con tu trabajo.
\section{Conceptos Fundamentales}
En esta sección se definen los conceptos clave que son relevantes para este trabajo. 

\subsection{Definición del Término A}
El término A se refiere a...

\subsection{Definición del Término B}
El término B, por otro lado, se entiende como...

\section{Teorías Relacionadas}
Esta sección presenta las teorías principales que fundamentan el estudio.
\chapter{Metodología}
Describe la metodología utilizada en tu investigación, incluyendo los métodos, herramientas y procedimientos.

\chapter{Resultados}
Presenta los resultados obtenidos durante tu investigación.

\chapter{Discusión}
Analiza los resultados y compara con otros estudios o referencias relevantes.

\chapter{Conclusiones y Recomendaciones}
Resume los hallazgos principales y ofrece recomendaciones futuras.

% Bibliografía
\addcontentsline{toc}{chapter}{Bibliografía}
\bibliographystyle{plain}
\begin{thebibliography}{9}
	\bibitem{1} Autor A, et al. Título del artículo. Revista, año.
\end{thebibliography}
\bibliography{referencias}

% Anexos (opcional)
\appendix
\chapter{Anexo A}
Incluye aquí cualquier información adicional como encuestas, gráficos, o tablas complementarias.

\end{document}